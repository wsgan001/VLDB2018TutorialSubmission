\documentclass{vldb}



%------------------------------------------------------------------------------- 
% Math packages
%------------------------------------------------------------------------------- 

% The "amsmath" package provides advanced math extensions.
\usepackage{amsmath}

% The "amssymb" package adds new symbols to be used in math mode.
\usepackage{amssymb}

% The "amsthm" package adds the "proof" environment and "theoremstyle" command.
% \usepackage{amsthm}

% The "faktor" package adds the "faktor" macro for variable substitution (i.e. vulgar fractions). This depends on the "amssymb" package.
\usepackage{faktor}

% The "semantic" package adds new macros for (PL-style) inference rules.
\usepackage[inference]{semantic}

%------------------------------------------------------------------------------- 
% Figure packages
%------------------------------------------------------------------------------- 

% The "fancyvrb" package provides advanced customization of verbatim environments, such as font families, numbering lines, box borders etc.
\usepackage{fancyvrb}

% The "graphicx" package allows including external graphic files.
\usepackage{graphicx}

% The "subfig" package allows multiple sub-figures within a single figure, where sub-figures can be separately captioned and labeled, e.g. Figure % 1.2(a). This is a replacement for the older "subfigure" package.
\usepackage{subfig}

% HACK: The caption package (included by the subfig package) requires a counter for ACM's copyright box.
\newcounter{copyrightbox}

% The "float" package allows the "H" option for figures, which places a float % at a precise location.
\usepackage{float}

% The "caption" package allows captions for figures that are not actually in a floating environment (e.g. framed environment).
\usepackage{caption}

% The "mdframed" package creates framed regions that can break across pages.
\usepackage{mdframed}

% The "algorithm2e" package provides keywords for typesetting algorithms. The "noend" option disables the printing of the "end" keywords. Use "algomargin" to decrease the margins for all algorithms.

% Kevin: To resolve conflict of algorithm2e with other packages, a common problem with ACM template.
% See http://ergodicthoughts.blogspot.com/2009/06/latex-too-many-s-algorithm2e.html
% \makeatletter
% \newif\if@restonecol
% \makeatother
% \let\algorithm\relax
% \let\endalgorithm\relax
% \usepackage[noend,boxed]{algorithm2e}
% \setlength{\algomargin}{0.5em}

%------------------------------------------------------------------------------- 
% Layout packages
%------------------------------------------------------------------------------- 

% The "multirow" package allows table cells to span more than one row.
\usepackage{multirow}

% The "balance" package allows columns of the last page to be of equal height.
\usepackage{balance}

% The "fixltx2e" package prevents two-column figures from being placed out-of-order wrt regular (one-column) figures.
\usepackage{fixltx2e}

% The "dblfloatfix" package allows two-column figures to be placed at the page's bottom.
% \usepackage{dblfloatfix}


%------------------------------------------------------------------------------- 
% Whitespace packages
%------------------------------------------------------------------------------- 

% The "savetrees" package saves space on a page.
% \usepackage[all=normal,paragraphs=tight,floats=tight,bibnotes=tight]{savetrees}
% \usepackage[all=normal,paragraphs=tight,floats=tight,bibnotes=tight,bibliography=tight]{savetrees}

% The "setspace" package allows changing the inter-line spacing to be a multiple of the default line spacing.
\usepackage{setspace}
%\setstretch{0.98}

% The "titlesec" package allows changing the whitespace around section headings.
% \usepackage[compact]{titlesec}

%------------------------------------------------------------------------------- 
% Font packages
%------------------------------------------------------------------------------- 

% The "beramono" package provides Bitstream Vera Mono, which has a bold typewritter fontface.
\usepackage[scaled]{beramono}
\usepackage[T1]{fontenc}

% The "courier" package provides Courier, which has a bold typewritter fontface.
%\usepackage{courier}


%------------------------------------------------------------------------------- 
% Misc packages
%------------------------------------------------------------------------------- 

% The "optional" package allows multiple versions of the document via optional text.
%\usepackage{optional}

% The "xstring" package allows switch/case conditionals.
\usepackage{xstring}

% The "xcolor" package allows colored text and backgrounds.
\usepackage[table]{xcolor}

% The "soul" package allows highlighting.
\usepackage{soul}

% The "ulem" package allows highlighting.
\usepackage[normalem]{ulem}

% The "tocloft" packages allows generating custom lists that are similar to table of contents, list of figures etc.
% \usepackage[subfigure]{tocloft}

% The "hyperref" package allows creating hyperlinks. Note that it must be the last package loaded, and will automatically includes the "url" package.
\usepackage{hyperref}

% The "hypcap" package fixes "hyperref" so that hyperlinks go to the top of a float (as opposed to its caption).
\usepackage[all]{hypcap}


\usepackage{color}

%------------------------------------------------------------------------------- 
% Whitespace
%------------------------------------------------------------------------------- 

% Adjust whitespace above and below captions
% \addtolength{\abovecaptionskip}{-5pt}
% \addtolength{\belowcaptionskip}{-9pt}

% Adjust whitespace before/after floats
%\setlength{\textfloatsep}{6pt plus 1.0pt minus 2.0pt}
%\setlength{\floatsep}{6pt plus 1.0pt minus 1.0pt}

%------------------------------------------------------------------------------- 
% Macros
%------------------------------------------------------------------------------- 

% Define our own compact enumerate
\newenvironment{compact_enum}
{\setlength{\leftmargini}{1em}
\begin{enumerate}
  \setlength{\labelsep}{.3em} 
  \setlength{\itemsep}{.4em}
  \setlength{\parskip}{0pt}
  \setlength{\parsep}{0pt}}
{\end{enumerate}}

% Define our own compact itemize
\newenvironment{compact_item}
{\setlength{\leftmargini}{1em}
\begin{itemize}
  \setlength{\labelsep}{.3em} 
  \setlength{\itemsep}{.4em}
  \setlength{\parskip}{0pt}
  \setlength{\parsep}{0pt}}
{\end{itemize}}

% \newtheorem{theorem}{Theorem}[section]
% \newtheorem{lemma}[theorem]{Lemma}
% \newtheorem{proposition}[theorem]{Proposition}
% \newtheorem{corollary}[theorem]{Corollary}

% \newenvironment{proof}[1][Proof]{\begin{trivlist}
% \item[\hskip \labelsep {\bfseries #1}]}{\end{trivlist}}
% \newenvironment{definition}[1][Definition]{\begin{trivlist}
% \item[\hskip \labelsep {\bfseries #1}]}{\end{trivlist}}
% \newenvironment{example}[1][Example]{\begin{trivlist}
% \item[\hskip \labelsep {\bfseries #1}]}{\end{trivlist}}
% \newenvironment{remark}[1][Remark]{\begin{trivlist}
% \item[\hskip \labelsep {\bfseries #1}]}{\end{trivlist}}

% \newcommand{\qed}{\nobreak \ifvmode \relax \else
%       \ifdim\lastskip<1.5em \hskip-\lastskip
%       \hskip1.5em plus0em minus0.5em \fi \nobreak
%       \vrule height0.75em width0.5em depth0.25em\fi}


%------------------------------------------------------------------------------- 
% Database symbols
%------------------------------------------------------------------------------- 

\def\join{$\bowtie$}
\def\ojoin{\setbox0=\hbox{$\bowtie$}%
  \rule[-.02ex]{.25em}{.4pt}\llap{\rule[\ht0]{.25em}{.4pt}}}
\def\leftouterjoin{\mathbin{\ojoin\mkern-5.8mu\bowtie}}
\def\rightouterjoin{\mathbin{\bowtie\mkern-5.8mu\ojoin}}
\def\fullouterjoin{\mathbin{\ojoin\mkern-5.8mu\bowtie\mkern-5.8mu\ojoin}}
\def\semijoin{\mbox{$\mathrel{\raise1pt\hbox{\vrule height5pt depth0pt\hskip-1.5pt$>$\hskip -2.5pt$<$}}$}}
\def\antisemijoin{\overline{\semijoin}}

%------------------------------------------------------------------------------- 
% Grammar symbols (BNFs and Tree Grammars) 
%------------------------------------------------------------------------------- 

% Formatting commands for tree grammars
\newcommand{\gn}[1]  {\textit{#1}}           % (N)on-terminal
\newcommand{\gt}[1]  {\texttt{\textbf{#1}}}  % (T)erminal
\newcommand{\gl}[1]  {\texttt{\textbf{#1}}}  % (L)iteral
\newcommand{\gs}[1]  {\textit{#1}}           % (S)pecial construct
\newcommand{\gob}[0] {[}                     % (O)ptional begin
\newcommand{\goe}[0] {]}                     % (O)ptional end
\newcommand{\gr}[0]  {...}                   % (R)epeat
\newcommand{\gp}[0]  {$\rightarrow$}         % (P)roduction rule
\newcommand{\gd}[0]  {$|$}                   % (D)isjunction

%------------------------------------------------------------------------------- 
% Misc symbols
%-------------------------------------------------------------------------------

% Undirected single quote mark
\chardef\singlequote=13




% http://tex.stackexchange.com/questions/32683/rotated-column-titles-in-tabular
\usepackage{adjustbox}
\newcolumntype{R}[2]{%
    >{\adjustbox{angle=#1,lap=\width-(#2)}\bgroup}%
    l%
    <{\egroup}%
}
\newcommand*\rot[2]{\multicolumn{1}{R{#1}{#2}}}% no optional argument here, please!

\usepackage{pifont}
\usepackage{wasysym}


\newcolumntype{B}{|@{~}r@{~}|@{~}l@{~}c@{~}l@{~}|}  % (B)NF
\newcolumntype{E}{@{}l@{~}c@{~}l@{}}                % (E)valuation Semantics
\newcolumntype{N}{@{}l@{}}                          % (N)arrow

\newcommand{\header}[1]{\multicolumn{1}{c|}{\textbf{#1}}}
\newcommand{\feature}[1]{\textbf{#1}:}

\usepackage{balance}
\vldbAuthors{A. Deutsch, Y. Papakonstantinou}
\vldbTitle{Graph Data Models, Query Languages and Programming Paradigms}
\vldbDOI{https://doi.org/TBD}

\begin{document}

% Copyright
%\setcopyright{acmcopyright}
%\setcopyright{acmlicensed}
%\setcopyright{rightsretained}
%\setcopyright{usgov}
%\setcopyright{usgovmixed}
%\setcopyright{cagov}
%\setcopyright{cagovmixed}

%\CopyrightYear{2016} 
%\isbn{978-1-4503-3531-7/16/06}
%\acmPrice{\$15.00}
%\doi{http://dx.doi.org/10.1145/2882903.2912573}
%\conferenceinfo{SIGMOD'16,}{June 26-July 01, 2016, San Francisco, CA, USA}


%------------------------------------------------------------------------------- 
% Macros
%------------------------------------------------------------------------------- 

% Set the version text flags on the "optional" package
%\setversiontextflags

% Redefine \_ for a shorter rule, so that BNF non-terminals with two (or more words) are compacted together
% http://tex.stackexchange.com/questions/4375/chardef-and-underscore
%\DeclareTextCommand{\_}{T1}{\leavevmode \kern.06em\vbox{\hrule width.4em}}

%\def\sugar{\cellcolor[gray]{0.8}}
\def\sugar{\rowcolor[gray]{0.85}}

\def\sql{A}
\def\cql{B}
\def\mongodb{C}
\def\nickel{D}
\def\aql{E}
\def\jaql{F}

\def\env{\mathrm{\Gamma}}

\def\yes{$\checkmark$}
\def\error{$\times$}
\def\true{\texttt{t}}
\def\false{\texttt{f}}
\def\nulloption{\texttt{n}}
\def\missing{\texttt{m}}
\def\counter{\texttt{c}}
\def\partial{\LEFTcircle}
\def\irrelevant{-}
\def\inconsistent{\ding{106}}
% \def\same{\cellcolor[gray]{0.8}}

\def\ground{G}
\def\collscan{S^C}
\def\outercollscan{O^C}
\def\tuplescan{S^T}
\def\outertuplescan{O^T}
\def\tuplenav{N^T}
\def\arraynav{N^A}
\def\constructcoll{C^C}
\def\constructtuple{C^T}
\def\applyplan{\alpha}
\def\corr{R}
\def\flat{F}

\def\evalto{\leadsto}
\def\group{f_\texttt{GROUP}}
\def\order{<^o}
%\def\order{f_\texttt{ORDER}}
\def\arr{\operatorname{array}}
\def\bag{\operatorname{bag}}
\def\tuple{\operatorname{tuple}}
\def\map{\operatorname{map}}
\def\fst{\operatorname{fst}}
\def\sub{\operatorname{sub}}
\def\sort{\operatorname{sort}}
\def\setop{\operatorname{set\_op}}
\def\union{\operatorname{union}}
\def\unionall{\operatorname{union\_all}}
\def\intersect{\operatorname{intersect}}
\def\intersectall{\operatorname{intersect\_all}}
\def\except{\operatorname{except}}
\def\exceptall{\operatorname{except\_all}}
\def\setopeq{\overset{@}{=}}

\newcommand{\inrecord}[1]   { {#1}_{\operatorname{in}} } % {\dot{#1}}
\newcommand{\outrecord}[1]  { {#1}_{\operatorname{out}} } % {\ddot{#1}}
\newcommand{\inbinding}[1]  { {#1}_{\operatorname{in}} }
\newcommand{\outbinding}[1] { {#1}_{\operatorname{out}} }

\newcommand{\id}[1]{{#1}^{id}}

\newcommand{\update}[2]{\gl{update}(#1,#2)}
\newcommand{\insertbag}[2]{\gl{insertinbag}(#1,#2)}
\newcommand{\inserttuple}[3]{\gl{inserinttuple}(#1,#2,#3)}
\newcommand{\append}[2]{\gl{append}(#1,#2)}
\newcommand{\delete}[1]{\gl{delete}(#1)}
\newcommand{\insertorder}[2]{\gl{insertorder}(#1,#2)}


\newcommand{\highlight}[1]{\noindent\textbf{#1:}}


\newcommand{\arem}[1]{{\begin{small}\color{blue}{#1\ --alin}\end{small}}}
\renewcommand{\arem}[1]{}
\newcommand{\yrem}[1]{{\begin{small}\color{red}{#1\ --yannis}\end{small}}}
\renewcommand{\yrem}[1]{}
\newcommand{\eat}[1]{}

\title{Graph Data Models, Query Languages and Programming Paradigms
\titlenote{Supported by NSF IIS129263, NSF SHB1237174, Informatica Inc. gift and Couchbase Inc. gift.}
}
\subtitle{[Tutorial Summary]}

\numberofauthors{2}

\author{
  \alignauthor Alin Deutsch\\
    \affaddr{UC San Diego}\\
    \email{deutsch@cs.ucsd.edu}
  \alignauthor Yannis Papakonstantinou\\
    \affaddr{UC San Diego}\\
    \email{yannis@cs.ucsd.edu}
}%



\maketitle

\begin{abstract}
  Numerous databases support semi-structured, schemaless and heterogeneous data, typically in the form of graphs (often restricted to trees and nested data). They also provide corresponding high-level query languages or graph-tailored programming paradigms.

The evolving query languages present multiple variations: Some are superficial syntactic ones, while other ones are genuine differences in modeling, language capabilities and semantics. Incompatibility with SQL presents a learning challenge for graph databases, while table orientation often leads to cumbersome syntactic/semantic structures that are contrary to graph data. Furthermore, the query languages often fall short of full-fledged semistructured and graph query language capabilities, when compared to the yardsticks set by prior academic efforts. 

We survey features, the designers' options and differences in the approaches taken by current systems. We cover both declarative query languages, whose semantics is
independent of the underlying model of computation, as well as languages with an operational semantics that is more tightly coupled with the model of computation.
For the declarative languages over both general graphs and tree-shaped graphs (as motivated by XML and the recent generation of nested formats, such as JSON and Parquet) 
we present SQL extensions that capture the essentials of such database systems.
More precisely, rather than presenting a single SQL extension, we present multiple configuration options whereas multiple possible (and different) semantics are formally captured by the multiple options that the language's semantic configuration options can take. We show how appropriate setting of the configuration options morphs the semantics into the semantics of multiple surveyed languages, hence providing a compact and formal tool to understand the essential semantic differences between different systems.

Finally we compare with prior nested and graph query languages (notably OQL, XQuery, Lorel, StruQL, PigLatin) and we transfer into the modern graph database context lessons from the semistructured query processing research of the 90s and 00s, combining them with insights on current graph databases.
%Again, the tutorial presents the algebras' fundamentals while it abstracts away modeling differences that are not applicable.
\end{abstract}

\begin{sloppypar}



%\begin{compact_item}
%\item \textbf{Duration:} The tutorial can be formatted to either 1.5 hours or 3 hours. We propose the \textbf{1.5 hour} format. See reasoning in Section~\ref{sec:duration}.
%\item \textbf{Target Audience:} Given the amount of activity on database applied R\&D and the large number of practitioners working with semistructured data (esp. JSON) in the Bay Area, the tutorial is addressed to both academic researchers and such industry parties. The tutorial is also addressed to the database researcher and/or database practitioner who is interested in NoSQL, SQL/JSON and (broadly) semistructured data querying, although she may not be knowledgeable on the topic(s) currently.
%\end{compact_item}

\eat{
\section{Introduction}
This is a summary of the SIGMOD 2016 tutorial. \textit{The reader is referred to \textbf{http://db.ucsd.edu/TutorialSIGMOD16/} for the complete material, which includes the full presented slideware and extended companion papers.}
}



%\begin{compact_item}
%\item \textbf{Duration:} The tutorial can be formatted to either 1.5 hours or 3 hours. We propose the \textbf{1.5 hour} format. See reasoning in Section~\ref{sec:duration}.
%\item \textbf{Target Audience:} Given the amount of activity on database applied R\&D and the large number of practitioners working with semistructured data (esp. JSON) in the Bay Area, the tutorial is addressed to both academic researchers and such industry parties. The tutorial is also addressed to the database researcher and/or database practitioner who is interested in NoSQL, SQL/JSON and (broadly) semistructured data querying, although she may not be knowledgeable on the topic(s) currently.
%\end{compact_item}

\section{Introduction}
This is a summary of the SIGMOD 2016 tutorial. \textit{The reader is referred to \textbf{http://db.ucsd.edu/TutorialSIGMOD16/} for the complete material, which includes the full presented slideware and extended companion papers.}


\section{Topic and Audience of This Tutorial}
Numerous databases promoted as SQL-on-Hadoop, NewSQL and NoSQL support Big Data applications. These databases generally support the 3Vs.% \cite{big-data-3v-oreilly-2012} 
(i) Volume: amount of data (ii) Velocity: speed of data in and out (iii) \emph{Variety}: semi-structured, schemaless and heterogeneous data, which is the focus of this tutorial. Due to the Variety requirement, many databases have adopted semi-structured data models, which are generally slightly different subsets of enriched JSON. The databases provide corresponding query languages. 

In addition to these {\em genuine JSON databases} that utilize variants of JSON as their data model, relational databases also provide special functions and language features for the support of JSON columns, often piggybacking on non-1NF (non first normal form) features that SQL acquired over the years. We refer to SQL databases with JSON support as {\em SQL/JSON databases}. 

In total, the tutorial's evolving companion survey 
%\cite{sqlpp-survey-2015} 
discusses the following as of April 2016 (and potentially more by the time the reader accesses it).

\begin{compact_enum}
\item Apache Hive \cite{hive-icde-2010} (description of Hive largely applicable to Cloudera Impala \cite{impala} also)
\item IBM Jaql \cite{jaql-pvldb-2011}
\item Apache Pig \cite{pig-sigmod-2008}
\item Apache Cassandra CQL \cite{cassandra-osr-2010}
\item MongoDB \cite{mongodb}
\item Couchbase's N1QL \cite{couchbase,couchbase-sigmod-2016}
\item JSONiq \cite{jsoniq-ieee-ic-2013}
\item AsterixDB's AQL \cite{asterixdb-dpd-2011-all-authors}
\item Google Big Query (aka Dremel \cite{dremel-pvldb-2010})
\item Mongo JDBC \cite{unityjdbc} (a JDBC driver provided by the UnityJDBC middleware for SQL-compliant access to MongoDB)
\item SQL-92 - as the anchor of SQL compliance
\end{compact_enum}

Myriads of developers and researchers currently use genuine JSON databases as well as SQL/JSON databases. Database builders and researchers work on expanding the databases' query language abilities. Both parties face the challenges described below regarding surveying and comparing models and query languages, past and present. This tutorial provides a deep understanding of the current data models and query languages of genuine JSON and SQL/JSON databases, hence enabling comparisons. 

The tutorial does not limit itself to the current status of JSON querying: The database builders and researchers need to draw lessons from the rich body of past research on nested \cite{nest-unnest-pods-1982,nested-relational-vldb-1988,nested-relational-workshop-lncs-1989}, object-oriented \cite{oql-dbpl-1989} and semistructured data models and querying \cite{xml-ql-computer-networks-1999,xml-query-language-survey-sigmod-record-2000,xquery-3.0-w3c-2013}, which have been a topic of intense database research: They were first researched in the form of labeled graphs in the mid-90s. Then semistructured data research boomed in the form of XML and its labeled tree abstraction. The current industrial boost to semistructured data emerged primarily from startups, often on the mobile and web space, that utilize Javascript and JSON.%
\footnote{Their JSON motivation is essentially the same one that the early semistructured research works had: flexible, schemaless data. Douglas Crockford's book ``Javascript: The Good Parts" characteristically muses about JSON that ``the less we need to agree on in order to interoperate, the more easily we can interoperate".
}
Many of the important language design issues of the first eras must be recalled in the new era of semistructured data.

Similarly, when it comes to implementations, we discuss set-at-a-time algebras for semistructured and object-oriented data, in the interest of transferring into the new space time-tested lessons on algebra-based relational query processing as well as lessons from set-at-a-time algebras from past research on nested relational algebra, OQL and XQuery \cite{GTP,XAT,OQL,SAL,DanaVLDB2004,Re06,MichielsTreePattern07,IoanaFQAS06,NicoleSerge,WiscMaier,Sartiani-Algebra,TAX,Enosys,Timber,NatixVLDBJ,Rainbow}. 

More broadly, we compare SQL (which we use as a baseline) with the recent crop of semistructured databases and connect the recent activity around JSON querying to the (much richer) past activity on nested relational, OQL and XML/labeled tree models and respective query languages and implementations. 

\section{Challenges in Comprehending the Space of Semistructured DBs}
A first challenge is that the evolving query languages have many variations. Some variations are due to superficial syntactic differences that simply create ``noise" when one tries to understand and compare systems. However, other variations are genuine differences in query language capabilities and semantics.

Indeed, the evolving query languages of both the genuine semistructured databases and the SQL/JSON databases fall short of full-fledged semi-structured query language capabilities.%
\footnote{Most genuine semistructured databases also fall significantly short of full-fledged SQL capabilities, which is not surprising since many commercial JSON databases started as key-value and document-oriented databases.}
The designers of the new query languages can gain by understanding and picking the salient features of past full-fledged \textit{declarative} query languages for non-relational data models: OQL \cite{oql-dbpl-1989}, the nested relational model \cite{nest-unnest-pods-1982,nested-relational-vldb-1988,nested-relational-workshop-lncs-1989}, 
XQuery, and other XML query languages \cite{xquery-3.0-w3c-2013,xml-ql-computer-networks-1999,xml-query-language-survey-sigmod-record-2000}.

Part of the confusion around current genuine JSON query languages is derived from the lack of compatibility with the well known SQL. In the interest of broadening the audience, this tutorial assumes that the audience is well-aware of SQL and the standard material of graduate textbooks on SQL system implementation. The tutorial does not assume knowledge of other query languages. Consequently we explain the JSON model and query languages as minimal extensions to SQL.S

Similarly, part of the confusion and the large semantics behind SQL/JSON is due to the retrofit of SQL for JSON columns, while certain limitations of SQL (such as the need of a schema) remain in place. Again, the tutorial does not require knowledge of non-1NF, often proprietary, features that have been added to SQL. Rather it only requires textbook SQL-92 language and teaches the non-1NF concepts.

A final challenge in understanding the new space of semistructured data is the lack of a succinct, mathematically clear, formal syntax and semantics by the vendors. 
%SQL/JSON vendors efforts towards a upgrading on SQL generate unecessarily large syntax and semantics.

In summary, the mentioned challenges and confusions hurt researchers and developers:

\begin{enumerate}
\item They inhibit a deep understanding of the capabilities and important idiosyncracies of the various query languages. Potential users can be lost in superficial details and miss fundamental points.
\item They impede progress towards declarative languages and systems for querying semi-structured data. Language designers and query processor implementors need to appreciate the available options, in order to proceed to well-designed fully-fledged languages and efficient implementations thereof. 
\end{enumerate}




\section{Challenges in Comprehending the Space of Graph and Nested Data DBs}
A first challenge is that the evolving semistructured query languages have many variations. (By ``semistructured" we refer to both graph and nested data.) Some variations are due to superficial syntactic differences that simply create ``noise" when one tries to understand and compare systems. However, other variations are genuine differences in query language capabilities and semantics.

Indeed, the evolving query languages of both the genuine semistructured databases and the SQL/JSON databases fall short of full-fledged semi-structured query language capabilities - as set by early academic efforts in the 90s.%
\footnote{Most semistructured databases also fall significantly short of full-fledged SQL capabilities.}
The designers of the new query languages can gain by understanding and picking the salient features of past full-fledged \textit{declarative} query languages for non-relational data models: OQL \cite{oql-dbpl-1989}, the nested relational model \cite{nest-unnest-pods-1982,nested-relational-vldb-1988,nested-relational-workshop-lncs-1989}, 
XQuery, and other XML query languages \cite{xquery-3.0-w3c-2013,xml-ql-computer-networks-1999,xml-query-language-survey-sigmod-record-2000}.

Part of the confusion around semistructured query languages is derived from the lack of compatibility with the well known SQL. In the interest of broadening the audience, this tutorial assumes that the audience is well-aware of SQL and the standard material of graduate textbooks on SQL system implementation. The tutorial does not assume knowledge of other query languages. Consequently we explain the JSON model and query languages as minimal extensions to SQL. In particular SQL-92, as it represents the well-supported common denominator of all SQL systems and corresponds to normalized databases. The tutorial does not require knowledge of non-1NF, often proprietary, features that have been added to SQL-92. Rather it only requires textbook SQL-92 language and teaches the non-1NF concepts, as well as the graph concepts.

A final challenge in understanding the new space of semistructured data is the lack of a succinct, mathematically clear, formal syntax and semantics by the vendors. 
%SQL/JSON vendors efforts towards a upgrading on SQL generate unecessarily large syntax and semantics.

In summary, the mentioned challenges and confusions hurt researchers and developers:

\begin{enumerate}
\item They inhibit a deep understanding of the capabilities and important idiosyncracies of the various query languages. Potential users can be lost in superficial details and miss fundamental points.
\item They impede progress towards declarative languages and systems for querying semi-structured data. Language designers and query processor implementors need to appreciate the available options, in order to proceed to well-designed fully-fledged languages and efficient implementations thereof. 
\end{enumerate}


\section{A systematic survey of model and language options and variations}

\noindent \textbf{Step 1: Extending the relational model and SQL for graphs} As discussed above, part of the confusion is derived from the lack of compatibility with the well-known, baseline SQL, which both researchers and practitioners generally understand.

Towards a uniform explanation of the large space of current and past systems, this tutorial reduces the declarative languages to (minimally extended) SQL.
We start by recalling the insight graphs can be modeled as relational databases by using appropriate vertex tables and edge tables. Vice versa, most relational databases can be viewed as graphs whose vertices are tuples and whose edges are key-foreign key pairs. We develop this analogy and show that multi-way SQL join queries correspond to fixed-length multi-hop path navigation. We then introduce introduce a series of minimal extensions that enhance the expressiveness of SQL towards reaching that of graph query languages.

One extension introduces controlled amounts of recursion via {\em path expressions} that specify reachability 
%constraints 
in the graph. We track the concept's evolution from early systems like OQL~\cite{oql-dbpl-1989},
Lorel~\cite{lorel}, WebSQL~\cite{websql}, StruQL~\cite{struql} via the standard XPath/XQuery~\cite{XQuery}
and the de facto standard Regular Path Queries (RPQs)~\cite{crpqs}, all the way to
contemporary languages such as Cypher~\cite{cypher} and Gremlin~\cite{gremlin}.
Path expressions may start with a variable, and multiple path expressions may appear in the \texttt{FROM} clause, potentially
correlated with previously defined variables of the same \texttt{FROM} clause. This correlation feature is ruled out by SQL-92 but it has been prominently present since OQL.

Additional extensions collaborate towards ensuring language compositionality, by enabling queries to output graphs.
A key enabler is the ability to invent fresh values to model the identities of newly constructed nodes and edges. This ability has its roots in object-oriented languages and its various incarnations can be invoked by the programmer either explicitly or implicitly.
We additionally demonstrate simultaneous construction of multiple linked tuples via each application of the \texttt{SELECT} clause.
Moreover, we show full compositionality, in the sense that subqueries can appear anywhere, potentially creating nested results when they appear in the \texttt{SELECT} clause.

Query languages for unrestricted graphs {\em annotated with scalar data} are compared via reduction to this extended SQL.
\arem{we need to make a high-level decision. SQL++ as is covers all the SQL/JSON dbs, but not the other graph databases. So we either formulate extensions to SQL++ (what shall we call
  the resulting language?), or we define another, more general language and mention SQL++ as sublanguage that is already in circulation. Yannis: Let's stay away of names, per phone call.}

We next shift attention to a highly important class of graphs: they are tree-shaped and annotated with non-1NF data.
This class is motivated by the plethora of semistructured databases in circulation today for formats such as JSON and Parquet (as well as their XML precursors). Explicit id invention is not necessary any more, as the construction of tree structures is accomplished by nested (sub)queries. We draw the parralels between the two classes of languages (semistructured and graph) highlighting the simplifications emerging in the semistructured case. We incorprorate into the discussion salient features of past full-fledged declarative query languages for non-relational data models: SQL non-1NF features (starting with SQL 2003), OQL,
% \cite{oql-dbpl-1989}, 
the nested relational model and query languages,
% \cite{nest-unnest-pods-1982,nested-relational-vldb-1988,nested-relational-workshop-lncs-1989} 
and XQuery (and other XML-based query languages).
% \cite{xquery-3.0-w3c-2013,xml-ql-computer-networks-1999,xml-query-language-survey-sigmod-record-2000}. 

An important issue, mostly discussed in the context of semistructured query languages (but fundamentally applying broadly to graphs) is the treatment of schema. The discussion includes language features that allow pivoting/unpivoting from schema to data and vice versa, thus enabling ``metadata'' inspection.

The corresponding plethora of query languages is classified also via reduction to an SQL extension, \emph{Configurable Graph SQL}.
Neglecting temporarily the ``configurable" aspect (discussed in Step~2), one may think of the presented language as an extension/modification of SQL-92 for graphs and semistructured data.

In this tutorial, a new student/researcher of graph and semistructured data, who missed the OQL and XQuery eras, will be able to absorb the essential teachings of OQL and XQuery while they are succinctly cast as a {\em minimally modified SQL}. We describe these modifications next, which will enable an audience member with SQL background to comprehend the fundamentals of the extension to genuine JSON databases with minimal effort.

After having taught Step 1, we will be able to show that multiple model and language differences are superficial syntactic differences.

\noindent \textbf{Step 2: Substantial Semantic Differences} However, not all differences are superficial. Furthermore, this tutorial does not suggest that the SQL extension (or some close descendant thereof) will become a standard and remove the many variations that are now found in this space. There is too much variation and legacy for such to happen. Yet, the language designers and researchers need to know now the design options that are available to them and the options that have been used by others, especially as it pertains to the handling of schema-less aspects (semantics for paths leading to nowhere, semantics for type mismatches, etc). 
Towards this goal the tutorial stresses the Configurable aspect of the presented SQL extension, which is essentially a guery language generator. Depending on the \textit{ configuration options} that are chosen for various features, different capabilities are assumed and different semantics emerge.

One particular example where configuration options capture differences concisely, is the behavior of paths of the various query languages in the absence of information. For example, consider a JSON object \gt{\{a:1, b:2\}} and a path that navigates into the absent path \gt{c}. Languages differ on what is the result of \gt{c}. Is it an error? Is it a special value? If it is a special value, how does it behave in other features of the query language? Is the query writer given control on what special value may emerge or whether an error will be thrown? A configuration option captures these differences precisely. 

By appropriate choices of configuration options, the Configurable SQL++ semantics morphs into the semantics of other query languages. Hence, the audience will be able to understand the essential differences between the various query languages, without being swamped by their superficial syntactic differences. Given the time constraints, the tutorial will present a few examples of issues and classifications, leaving the complete surveying for an online survey that the authors will have set up.  

We expect that some of the results listed in the feature matrices describing configuration options will change in the next years as the space evolves rapidly.
Despite the forthcoming changes, we expect the configuration-based aspect of our tutorial to remain a standing tool in understanding the space, since by understanding each database's capabilities in terms of applicable options, the reader can focus on the fundamental differences of the databases. 

\noindent \textbf{Step 3: Additional Features} We will also discuss features, many of which coming from XQuery, that have not been captured by configuration options. These will prompt a more open-ended discussion of language designs and trade-offs. A notable one is type coercion - the approaches and the pros and cons.

\noindent \textbf{Step 4: Languages with Operational Semantics Tied to the Computation Model}
Recently we have witnessed a trend towards development of high-level graph query languages that are deliberately not purely declarative, instead featuring an operational semantics tied to the
underlying computation model, which is typically a Bulk Synchronous Parallel (BSP) instance presented as a variation of the  Map-Reduce programming paradigm.
Prominent examples are Gremlin and GSQL. We discuss this class of queries.


\noindent \textbf{Open Issues} Finally, we emphasize open issues in the expansion from structured to semistructured querying and briefly discuss interoperability challenges induced by language differences.




\section{Other Data}
\label{sec:duration}

\noindent \textbf{Prerequisite Knowledge} 
\label{sec:audience}
The attendees must have solid knowledge of SQL-92, since it is the baseline upon which the semistructured aspects are then added. Also solid knowledge of relational algebra. Knowledge of SQL-2003 and/or XQuery are a plus but not required.

\noindent \textbf{Relevant Aspects of Presenter's Bio} 
Yannis Papakonstantinou is a Professor of Computer Science and Engineering at UCSD. A common theme of his research is the extension of database platforms and query processors beyond centralized relational databases and into semistructured databases, integrated views of distributed databases and web services, textual data and queries involving keyword search. His research has received more than 12,500 citations, according to Google Scholar, most of which refer to his work on semistructured data, semistructured query processing and related middleware.
In addition to his academic activity in middleware, semistructured data and query processing, Yannis was the Chief Scientist of Enosys Software, which built and commercialized an early Enterprise Information Integration platform for structured and semistructured data, utilizing XML and XQuery. The Enosys Software was OEM'd and sold under the BEA Liquid Data and BEA Aqualogic brand names, eventually acquired in 2003 by BEA Systems.\\

Alin Deutsch is a Professor of Computer Science and Engineering at UCSD. His research pertinent to this tutorial has centered around the design and optimization of semi-structured query languages
and on programming paradigms for graph data analytics. Alin is the recipient of a PODS Test of Time Award, a SIGMOD Top-3 Best Paper Award, an Alfred F. Sloan fellowship, and an NSF Career award.
Alin has recently served as Senior Scientist of TigerGraph Inc., a start-up that offers an engine capable of real-time analytics on web-scale graph data, expressed in its own query language.

\arem{candidate for removal, we can add acks back into the camera-ready copy''}

\noindent \textbf{Acknowledgments} 
Yannis Papakonstantinou thanks his coauthors of \cite{sqlpp-extended-corr-2015} and \cite{sqlpp-survey-2015}, on which most of this tutorial presentation is based on.

Many of the language-related aspects of the proposed tutorial have been presented by Yannis Papakonstantinou during the past year at talks at the following companies and universities that develop NoSQL databases,  SQL-on-Hadoop databases or middleware that operates over them: Amazon, Couchbase, MapR, Informatica (which develops middleware for data warehousing and virtual databases over SQL, NoSQL, Hadoop-on-SQL and NewSQL databases), Pivotal. The above talks led to very useful feedback.


\end{sloppypar}


% Bibliography
\bibliographystyle{abbrv}
\bibliography{main,lot}

\end{document}
