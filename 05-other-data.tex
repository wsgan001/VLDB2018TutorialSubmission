

\section{Other Data}
\label{sec:duration}

\noindent \textbf{Prerequisite Knowledge} 
\label{sec:audience}
The attendees must have solid knowledge of SQL-92, since it is the baseline upon which the semistructured aspects are then added. Also solid knowledge of relational algebra. Knowledge of SQL-2003 and/or XQuery are a plus but not required.

\noindent \textbf{Relevant Aspects of Presenter's Bio} 
Yannis Papakonstantinou is a Professor of Computer Science and Engineering at UCSD. A common theme of his research is the extension of database platforms and query processors beyond centralized relational databases and into semistructured databases, integrated views of distributed databases and web services, textual data and queries involving keyword search. His research has received more than 12,500 citations, according to Google Scholar, most of which refer to his work on semistructured data, semistructured query processing and related middleware.
In addition to his academic activity in middleware, semistructured data and query processing, Yannis was the Chief Scientist of Enosys Software, which built and commercialized an early Enterprise Information Integration platform for structured and semistructured data, utilizing XML and XQuery. The Enosys Software was OEM'd and sold under the BEA Liquid Data and BEA Aqualogic brand names, eventually acquired in 2003 by BEA Systems.\\

Alin Deutsch is a Professor of Computer Science and Engineering at UCSD. His research pertinent to this tutorial has centered around the design and optimization of semi-structured query languages
and on programming paradigms for graph data analytics. Alin is the recipient of a PODS Test of Time Award, a SIGMOD Top-3 Best Paper Award, an Alfred F. Sloan fellowship, and an NSF Career award.
Alin has recently served as Senior Scientist of TigerGraph Inc., a start-up that offers an engine capable of real-time analytics on web-scale graph data, expressed in its own query language.

\arem{candidate for removal, we can add acks back into the camera-ready copy''}

\noindent \textbf{Acknowledgments} 
Yannis Papakonstantinou thanks his coauthors of \cite{sqlpp-extended-corr-2015} and \cite{sqlpp-survey-2015}, on which most of this tutorial presentation is based on.

Many of the language-related aspects of the proposed tutorial have been presented by Yannis Papakonstantinou during the past year at talks at the following companies and universities that develop NoSQL databases,  SQL-on-Hadoop databases or middleware that operates over them: Amazon, Couchbase, MapR, Informatica (which develops middleware for data warehousing and virtual databases over SQL, NoSQL, Hadoop-on-SQL and NewSQL databases), Pivotal. The above talks led to very useful feedback.
