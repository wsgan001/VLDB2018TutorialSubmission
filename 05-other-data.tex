

\section{Other Data}
\label{sec:duration}

\noindent \textbf{Prerequisite Knowledge} 
\label{sec:audience}
The attendees must have solid knowledge of SQL-92, since it is the baseline upon which the semistructured aspects are then added. Also solid knowledge of relational algebra. Knowledge of SQL-2003 and/or XPath are a plus but not required.

\smallskip
\noindent \textbf{Relevant Aspects of Presenters' Bio} 

\textbf{Alin Deutsch} is a Professor of Computer Science and Engineering at UCSD. His research pertinent to this tutorial centers around the design and optimization of semi-structured query languages
and on programming paradigms for graph data analytics. Alin is the recipient of an ACM PODS Test of Time Award, an ACM SIGMOD Top-3 Best Paper Award (together with tutorial co-author Yannis Papakonstantinou), an Alfred P. Sloan fellowship, and an NSF Career award.
He has served as Senior Scientist of TigerGraph Inc., a start-up that offers an engine capable of real-time analytics on web-scale graph data.
The analytic tasks are  expressed in a high-level graph query language called GSQL, in whose design Alin was involved.

\textbf{Yannis Papakonstantinou} is a Professor of Computer Science and Engineering at UCSD. A common theme of his research is the extension of database platforms and query processors beyond centralized relational databases and into semistructured databases, integrated views of distributed databases and web services, textual data and queries involving keyword search. His research has received more than 14,500 citations, according to Google Scholar, most of which refer to his work on semistructured data, semistructured query processing and related middleware.
In addition to his academic activity in middleware, semistructured data and query processing, Yannis was the Chief Scientist of Enosys Software, which built and commercialized an early Enterprise Information Integration platform for structured and semistructured data, utilizing XML and XQuery. The Enosys Software was OEM'd and sold under the BEA Liquid Data and BEA Aqualogic brand names. Yannis currently serves also as a consultant to Amazon Web Services.

